\documentclass[runningheads,a4paper]{llncs}
\usepackage{amssymb}
\usepackage{amsmath}
\usepackage{subfig}
\setcounter{tocdepth}{3}
\usepackage{graphicx}
%\usepackage{titlesec}
\usepackage[hidelinks]{hyperref}

\usepackage[english, spanish]{babel}
%\usepackage[utf8]{inputenc}
\usepackage{url}
\urldef{\mailsa}\path|https://github.com/Daroce1012/DAA-problems/tree/Tito_el_corrupto|    
\newcommand{\keywords}[1]{\par\addvspace\baselineskip
	\noindent\keywordname\enspace\ignorespaces#1}


\begin{document}
	
	\mainmatter 
	
	\tableofcontents
	\newpage
	\title{Proyecto de \\Dise\~no y An\'alisis de Algoritmos}
	
	%titulo del problema
	\titlerunning{Dise\~no y An\'alisis de Algoritmos}
	
	
	\author{Belsai Arango Hern\'andez \\ Daniela Rodr\'iguez Cepero }
	%
	\authorrunning{Belsai Arango Hern'andez y Daniela Rodr\'iguez Cepero }
	
	\institute{Universidad de La Habana,\\
		San L\'azaro y L. Plaza de la Revoluci\'on, La Habana, Cuba\\
		\mailsa\\
		\url{http://www.uh.cu}}
	
	
	\maketitle
	
	
	
\section{Problema}

	{\selectlanguage{spanish}
\begin{center}
{\large\bf Tito el corrupto}\\
\end{center}
Tito se di\'o cuenta de que la carrera de computaci\'on estaba acabando con \'el y
un d\'ia decidi\'o darle un cambio radical a su vida. Comenz\'o a estudiar Ingenier\'ia
Industrial. Luego de unos a\~nos de fiesta, logr\'o finalmente conseguir su t\'itulo
de ingeniero. Luego de otros tantos a\~nos ejerciendo sus estudios (¿?), consigui\'o
ponerse a la cabeza de un gran proyecto de construcci\'on de carreteras.\\

La zona en la que debe trabajar tiene $n$ ciudades con $m$ posibles carreteras
a construir entre ellas. Cada ciudad que sea incluida en el proyecto aportar\'a $a_i$
d\'olares al proyecto, mientras que cada carretera tiene un costo de $w_i$ d\'olares.
Si una carretera se incluye en el proyecto, las ciudades unidas por esta tambi\'en
deben incluirse.\\

El problema estar\'ia en que Tito quiere utilizar una de las habilidades que
aprendi\'o en sus a\~nos de estudio, la de la malversaci\'on de fondos. Todo el dinero
necesario para el proyecto que no sea un aporte de alguna ciudad, lo proveer\'ia
el pa\'is y pasar\'ia por manos de Tito. El dinero aportado por las ciudades no
pasar\'ia por sus manos. Tito quiere maximizar la cantidad de dinero que pasa
por \'el, para poder hacer su magia. Ayude a Tito a seleccionar el conjunto de
carreteras a incluir en el proyecto para lograr su objetivo

\newpage
\begin{abstract}
 Se tienen $n$ ciudades y $m$ posibles carreteras a construir, cada ciudad que sea incluida en el proyecto aportar\'a $a_i$ d\'olares y la construcci\'on de la carretera cuesta $w_i$ d\'olares. Todo el dinero necesario para el proyecto que no sea un aporte de alguna ciudad, lo proveer\'ia el pa\'is y entonces pasar\'ia a manos de Tito. Se quiere seleccionar un conjunto de carreteras a incluir en el proyecto que maximice la cantidad de dinero que pasa
por Tito. 

\end{abstract}

\section{Soluci\'on}

Tenemos que elegir un subconjunto de carreteras y un subconjunto de ciudades, de modo tal que si se elige una carretera, tambi\'en se elijan todas las ciudades de los que depende este proyecto, y tenemos que maximizar la diferencia entre la suma de los costos de las carreteras elegidas(cantidad de dinero que pasa por Tito) y la suma de los aportes de las ciudades unidas a estas(cantidad de dinero que no pasa por Tito), para as\'i maximizar la cantidad de dinero que pasa por Tito.\\

Es bastante inconveniente que por los carreteras obtengamos dinero y por las ciudades que est\'an unidas a esta se pierdan ganancias, as\'i que supongamos que obtenemos dinero del pa\'is por adelantado  por las carreteras y que tenemos que devolver el dinero, si construirla no nos genera ganancias. Estas 2 formulaciones son claramente equivalentes, pero la \'ultima es m\'as f\'acil de modelar.\\

Una forma sencilla ser\'ia crear un grafo con n+m+2 v\'ertices, donde cada v\'ertice representa la fuente(S), el sumidero(T), una carretera o una ciudad. Notaremos a la i-\'esima carretera como $W_i$ y la i-\'esima ciudad como $A_i$. Luego, se agregan aristas de peso $w_i$ entre la fuente y la i-\'esima carretera, aristas de peso $a_i$ entre la i-\'esima ciudad y el receptor, como el dinero que pasa por Tito de las carreteras dependen de las ciudades a las que est\'a unida, para representar esta restricci\'on se agrega una arista de peso $\infty$ entre la carretera y las ciudades que est\'an unidas a esta. Por \'ultimo aplicar un corte m\'inimo sobre este grafo.\\

Si se corta la arista entre la fuente y la i-\'esima carretera, entonces tenemos que devolver el dinero para la i-\'esima carretera. Si se corta la arista entre la i-\'esima ciudad y el receptor, entonces podemos valorar la construcci\'on de las carreteras unidas a la i-\'esima ciudad. Es f\'acil observar que para todas las dependencias entre una carretera y una ciudad, uniremos las ciudades necesarias o abandonaremos el proyecto ya que ser\'ia imposible cortar la arista de peso $\infty$ entre ellos.\\

Luego de aplicar el flujo m\'aximo con el corte antes mencionado nos quedar\'iamos con una red en donde las aristas que est\'an presentes representan las aristas no saturadas, y las que est\'an conectadas al receptor ser\'ian las ciudades cuyo aporte es mayor que el costo de la carretera a la que est\'a unida y por tanto la contrucci\'on de esta no le permitir\'ia a Tito lograr su objetivo. Entonces el conjunto de soluci\'on ser\'ia el de las carreteras que no est\'en unidas a estas ciudades.

\section{Correctitud}

Como problema requiere que se seleccione un conjunto de carreteras y ciudades unidas a estas de tal manera que nos quede el máximo número de beneficios.
Digamos que inicialmente tenemos todos los ingresos de los proyectos. Así que nuestro ingreso total (inicialmente) es igual a $\sum_{i=1}R(w_i)$.

Ahora, como cuando elegimos una carretera, se unen las ciudades correspondientes. Supongamos que las carreteras que no hemos seleccionado están en un conjunto P y las ciudades que nos hacen perder dinero están en un conjunto A. Entonces, definitivamente podemos escribir algo como esto:\\
max\{ingresos\} = $\sum_iR(w_i)-\sum_iR(Pi)-\sum_iC(Ai)$.\\

Podemos ver que la primera de las tres sumas en el lado derecho de la ecuaci\'on es constante. Ahora, si podemos minimizar las dos \'ultimas sumas, podemos obtener los ingresos m\'aximos.\\

Entonces queremos minimizar $\sum_iR(Pi)+\sum_iC(Ai)$.\\ 

As\'i que construimos un grafo donde cada carretera est\'a conectado a la fuente S con capacidad $R(w_i)$ y cada ciudad est\'a conectada al receptor T con capacidad $C(a_i)$. Y como las carreteras dependen de las ciudades que esta\'an unidas a esta, agregamos aristas desde las carreteras hacia las ciudades de las que depende con capacidad $\infty$ para garantizar que si se toma una carretera, también se toman las ciudades requeridas correspondientes.

Finalmente, calculamos el corte m\'inimo del grafo construido.\\

El corte sobre la red de flujo $G$ con flujo m\'aximo $f^*$ es de capacidad m\'inima.
	\textbf{Demostraci\'on:}
	\begin{itemize} 
		\item[]  Sea el corte $(S,T)$ formado por $S = \{s,v_1,v_2,...,v_k\} $ y $T = \{v_{k+1},v_{k+2},...,v_{n-2},t\} $ donde todo v\'ertice $v_i \in S$ es alcanzable por $s$ a trav\'es de un camino en la red residual $G_f$ y todo v\'ertice en $T$ ser\'ian los no alcanzables por $s$ bajo este criterio. Ambos conjuntos son no vac\'ios, pues a $S$ al menos pertenece el v\'ertice $s$ y a $T$ pertenece el v\'ertice $t$, pues si no perteneciera significa que existe un camino de $s$ hacia $t$ en $G_f$ lo que implica que ser\'ia un camino aumentativo y ser\'ia posible incrementar el flujo utilizando dicho camino lo cual entra en contradicci\'on con el hecho de que $f^*$ es un flujo m\'aximo en $G$.
		
		Luego, toda arista $< u,v >$ que cruza el corte $(S,T)$ de $S$ hacia $T$, donde $u \in S$ y $v \in T$ cumple que $f(< u,v >) = c(< u,v >)$, pues si existiese $< u,v >$ que cruza el corte tal que $f(< u,v >) < c(< u,v >)$ entonces la arista $< u,v >$ no estar\'ia saturada y existiera en $G_f$ por lo que el camino de $s$ hacia $u$ de aristas no saturadas, adicionando la arista $< u,v >$ forman un camino $s$ hacia $v$ en $G_f$ lo cual es una contradicci\'on pues $v \in T$. 
		
		Adem\'as toda arista $< v,u >$ que cruza el corte desde $T$ hacia $S$ donde $u \in S$ y $v \in T$ cumple que $f(< u,v >) = 0$, pues si existiese $< v,u >$ que cruza el corte tal que $f(< v,u >) > 0$ entonces dicha arista da origen a una arista inversa $< u,v >$ en $G_f$, como $u \in S$, entonces existe un camino desde $s$ hacia $u$ en $G_f$ que al unirse con la arista $< u,v >$ dar\'ia lugar a un camino de $s$ hacia $v$ en $G_f$ lo cual es una contradicci\'on con el hecho de que $v \in T$.
		
		Por lo tanto todas las aristas que salen de $S$ est\'an saturadas y todas las que entran tienen flujo 0. Demostremos entonces que la capacidad del corte $(S,T)$ es igual al valor del flujo m\'aximo $f^*$ en $G$. Sabemos que $|f^*| = f(S,T)$ por el $Lema$ 26.4 visto en conferencia. Luego: 
		
		$|f^*| = f(S,T) = \sum\limits_{u\in S}\sum\limits_{v\in T}f(u,v)-\sum\limits_{u\in S}\sum\limits_{v\in T}f(v,u)$ 
		
		Pero como demostramos anteriormente que $f(u,v)=c(u,v)$ y $f(v,u)=0$ donde $u \in S$ y $v \in T$ entonces:
		
		$|f^*| = f(S,T) = \sum\limits_{u\in S}\sum\limits_{v\in T}c(u,v)-\sum\limits_{u\in S}\sum\limits_{v\in T}0 = \sum\limits_{u\in S}\sum\limits_{v\in T}c(u,v) = C(S,T)$
		
		Luego la capacidad del corte $(S,T)$ es igual al valor del flujo m\'aximo $f^*$ en $G$ por lo que su capacidad es m\'inima ya que $|f^*|$ es una cota m\'inima de la capacidad de todo corte por $Corolario$ 26.5 visto en conferencias en EDA.  
	\end{itemize}   
	
Para hayar el flujo m\'aximo con corte m\'inimo se utiliza el algoritmo Edmonds-Karp que genera una red residual en donde las aristas que est\'an presentes representan las aristas no saturadas, y las que est\'an conectadas al receptor ser\'ian las ciudades cuyo aporte es mayor que el costo de la carretera a la que est\'a unida y por tanto la contrucci\'on de esta no le permitir\'ia a Tito lograr su objetivo. Entonces el conjunto de soluci\'on ser\'ia el de las carreteras que no est\'en unidas a estas ciudades.\\

\newpage
\section{Complejidad Temporal}
El algoritmo Edmonds-Karp se ejecuta $O(|V|*|E|)$ veces (ver demostraci\'on en I. to A. 3ra. Edición Cap. 26)
Pero sabemos adem\'as, que cada iteraci\'on toma tiempo $ O(|E|)$ ya que el camino aumentativo se halla utilizando un BFS (BFS es $O(|V|+|E|)$ pero en la red de flujo, $|V|$ es $O(|E|)$). El resto de las operaciones, o sea actualizar la Red Residual y el flujo que se va calculando, se pueden hacer tambi\'en en ese mismo orden de tiempo
Por tanto,
el tiempo de ejecuci\'on del algoritmo es $O(|V|*{|E|}^2)$.


\end{document}
