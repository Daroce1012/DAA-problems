\documentclass[runningheads,a4paper]{llncs}
\usepackage{amssymb}
\usepackage{amsmath}
\usepackage{subfig}
\setcounter{tocdepth}{3}
\usepackage{graphicx}
%\usepackage{titlesec}
\usepackage[hidelinks]{hyperref}

\usepackage[english, spanish]{babel}
%\usepackage[utf8]{inputenc}
\usepackage{url}
\urldef{\mailsa}\path|https://github.com/Daroce1012/DAA-problems/tree/Tito_el_corrupto|    
\newcommand{\keywords}[1]{\par\addvspace\baselineskip
	\noindent\keywordname\enspace\ignorespaces#1}


\begin{document}
	
	\mainmatter 
	
	\tableofcontents
	\newpage
	\title{Proyecto de \\Dise\~no y An\'alisis de Algoritmos}
	
	%titulo del problema
	\titlerunning{Dise\~no y An\'alisis de Algoritmos}
	
	
	\author{Belsai Arango Hern\'andez \\ Daniela Rodr\'iguez Cepero }
	%
	\authorrunning{Belsai Arango Hern'andez y Daniela Rodr\'iguez Cepero }
	
	\institute{Universidad de La Habana,\\
		San L\'azaro y L. Plaza de la Revoluci\'on, La Habana, Cuba\\
		\mailsa\\
		\url{http://www.uh.cu}}
	
	
	\maketitle
	
	
	
\section{Problema}

	{\selectlanguage{spanish}
\begin{center}
{\large\bf Tito el corrupto}\\
\end{center}
Tito se di\'o cuenta de que la carrera de computaci\'on estaba acabando con \'el y
un d\'ia decidi\'o darle un cambio radical a su vida. Comenz\'o a estudiar Ingenier\'ia
Industrial. Luego de unos a\~nos de fiesta, logr\'o finalmente conseguir su t\'itulo
de ingeniero. Luego de otros tantos a\~nos ejerciendo sus estudios (¿?), consigui\'o
ponerse a la cabeza de un gran proyecto de construcci\'on de carreteras.\\

La zona en la que debe trabajar tiene $n$ ciudades con $m$ posibles carreteras
a construir entre ellas. Cada ciudad que sea incluida en el proyecto aportar\'a $a_i$
d\'olares al proyecto, mientras que cada carretera tiene un costo de $w_i$ d\'olares.
Si una carretera se incluye en el proyecto, las ciudades unidas por esta tambi\'en
deben incluirse.\\

El problema estar\'ia en que Tito quiere utilizar una de las habilidades que
aprendi\'o en sus a\~nos de estudio, la de la malversaci\'on de fondos. Todo el dinero
necesario para el proyecto que no sea un aporte de alguna ciudad, lo proveer\'ia
el pa\'is y pasar\'ia por manos de Tito. El dinero aportado por las ciudades no
pasar\'ia por sus manos. Tito quiere maximizar la cantidad de dinero que pasa
por \'el, para poder hacer su magia. Ayude a Tito a seleccionar el conjunto de
carreteras a incluir en el proyecto para lograr su objetivo

\newpage
\begin{abstract}
 Se tienen $n$ ciudades y $m$ posibles carreteras a construir, cada ciudad que sea incluida en el proyecto aportar\'a $a_i$ d\'olares y la construcci\'on de la carretera cuesta $w_i$ d\'olares. Todo el dinero necesario para el proyecto que no sea un aporte de alguna ciudad, lo proveer\'ia el pa\'is y entonces pasar\'ia a manos de Tito. Se quiere seleccionar un conjunto de carreteras a incluir en el proyecto que maximice la cantidad de dinero que pasa
por Tito. 

\end{abstract}

\section{Soluci\'on}

Tenemos que elegir un subconjunto de carreteras y un subconjunto de ciudades, de modo tal que si se elige una carretera, tambi\'en se elijan todas las ciudades de los que depende este proyecto, y tenemos que maximizar la diferencia entre la suma de los costos de las carreteras elegidas(cantidad de dinero que pasa por Tito) y la suma de los aportes de las ciudades unidas a estas(cantidad de dinero que no pasa por Tito), para as\'i maximizar la cantidad de dinero que pasa por Tito.\\

Es bastante inconveniente que por los carreteras obtengamos dinero y por las ciudades que est\'an unidas a esta se pierdan ganancias, as\'i que supongamos que obtenemos dinero del pa\'is por adelantado  por las carreteras y que tenemos que devolver el dinero, si construirla no nos genera ganancias. Estas 2 formulaciones son claramente equivalentes, pero la \'ultima es m\'as f\'acil de modelar.\\

Una forma sencilla ser\'ia crear un grafo con n+m+2 v\'ertices, donde cada v\'ertice representa la fuente(S), el sumidero(T), una carretera o una ciudad. Notaremos a la i-\'esima carretera como $W_i$ y la i-\'esima ciudad como $A_i$. Luego, se agregan aristas de peso $w_i$ entre la fuente y la i-\'esima carretera, aristas de peso $a_i$ entre la i-\'esima ciudad y el receptor, como el dinero que pasa por Tito de las carreteras dependen de las ciudades a las que est\'a unida, para representar esta restricci\'on se agrega una arista de peso $\infty$ entre la carretera y las ciudades que est\'an unidas a esta. Por \'ultimo aplicar un corte m\'inimo sobre este grafo.\\

Si se corta la arista entre la fuente y la i-\'esima carretera, entonces tenemos que devolver el dinero para la i-\'esima carretera. Si se corta la arista entre la i-\'esima ciudad y el receptor, entonces podemos valorar la construcci\'on de las carreteras unidas a la i-\'esima ciudad. Es f\'acil observar que para todas las dependencias entre una carretera y una ciudad, uniremos las ciudades necesarias o abandonaremos el proyecto ya que ser\'ia imposible cortar la arista de peso $\infty$ entre ellos.\\

Luego de aplicar el flujo m\'aximo con el corte antes mencionado nos quedar\'iamos con una red en donde las aristas que est\'an presentes representan las aristas no saturadas, y las que est\'an conectadas al receptor ser\'ian las ciudades cuyo aporte es mayor que el costo de la carretera a la que est\'a unida y por tanto la contrucci\'on de esta no le permitir\'ia a Tito lograr su objetivo. Entonces el conjunto de soluci\'on ser\'ia el de las carreteras que no est\'en unidas a estas ciudades.

\section{Correctitud}

Como problema requiere que se seleccione un conjunto de carreteras y ciudades unidas a estas de tal manera que nos quede el máximo número de beneficios.
Digamos que inicialmente tenemos todos los ingresos de los proyectos. Así que nuestro ingreso total (inicialmente) es igual a $\sum_{i=1}R(w_i)$.

Ahora, como cuando elegimos una carretera, se unen las ciudades correspondientes. Supongamos que las carreteras que no hemos seleccionado están en un conjunto P y las ciudades que nos hacen perder dinero están en un conjunto M. Entonces, definitivamente podemos escribir algo como esto:\\
max\{ingresos\} = $\sum_iR(w_i)-\sum_iR(Pi)-\sum_iC(Mi)$.\\

Podemos ver que la primera de las tres sumas en el lado derecho de la ecuaci\'on es constante. Ahora, si podemos minimizar las dos \'ultimas sumas, podemos obtener los ingresos m\'aximos.\\

Entonces queremos minimizar $\sum_iR(Pi)+\sum_iC(Mi)$.\\ 

Así que construimos un gráfico donde cada proyecto está conectado a la fuente S con capacidad R(pi) y cada máquina está conectada al sumidero T con capacidad C(mi). Y en caso de que un proyecto en particular necesite una cierta cantidad de máquinas, agregamos bordes de cada proyecto a las máquinas que necesita con capacidad ∞ para garantizar que si se toma un proyecto, también se toman las máquinas requeridas correspondientes.

Finalmente, calculamos el corte mínimo del gráfico construido

\newpage
\section{Complejidad Temporal}
El algoritmo lee la entrada del problema e itera hasta $n$ (cantidad de ciudades ubicadas en $X0$) para colocar las coordenadas en una arreglo $a$. El orden de realizar dicha operaci\'on es $O(n)$. Luego ordena las coordenadas utilizando el algoritmo de ordenaci\'on de python que tiene una complejidad temporal de $O(n \log n)$. Para colocar en la primera posici\'on de la lista un elemento se emple\'o dos veces el reverse() de las listas de python lo cual es $O(n)$. Para ofrecer la soluci\'on se realizan dos recorridos, el primero desde $k$ hasta $n$, y el segundo desde 1 hasta $k$, en cada uno se realiza una cantidad constante de operaciones matem\'aticas elementales $O(1)$, como $0 \leq k \leq n$, la complejidad temporal ser\'a en el orden de $O(n)$. Por regla de la suma la complejidad temporal del algoritmo es de $O(n \log n)$. 

\end{document}
