\documentclass[runningheads,a4paper]{llncs}
\usepackage{amssymb}
\usepackage{amsmath}
\usepackage{subfig}
\setcounter{tocdepth}{3}
\usepackage{graphicx}
%\usepackage{titlesec}
\usepackage[hidelinks]{hyperref}
\usepackage{listings}

\usepackage[english, spanish]{babel}
%\usepackage[utf8]{inputenc}
\usepackage{url}
\urldef{\mailsa}\path|https://github.com/Daroce1012/DAA-problems/tree/Tito_el_corrupto|    
\newcommand{\keywords}[1]{\par\addvspace\baselineskip
	\noindent\keywordname\enspace\ignorespaces#1}


\begin{document}
	
	\mainmatter 
	
	\tableofcontents
	\newpage
	\title{Proyecto de \\Dise\~no y An\'alisis de Algoritmos}
	
	%titulo del problema
	\titlerunning{Dise\~no y An\'alisis de Algoritmos}
	
	
	\author{Belsai Arango Hern\'andez \\ Daniela Rodr\'iguez Cepero }
	%
	\authorrunning{Belsai Arango Hern'andez y Daniela Rodr\'iguez Cepero }
	
	\institute{Universidad de La Habana,\\
		San L\'azaro y L. Plaza de la Revoluci\'on, La Habana, Cuba\\
		\mailsa\\
		\url{http://www.uh.cu}}
	
	
	\maketitle
	
	
	
\section{Problema}

	{\selectlanguage{spanish}
\begin{center}
{\large\bf Tito el corrupto}\\
\end{center}
Tito se di\'o cuenta de que la carrera de computaci\'on estaba acabando con \'el y
un d\'ia decidi\'o darle un cambio radical a su vida. Comenz\'o a estudiar Ingenier\'ia
Industrial. Luego de unos a\~nos de fiesta, logr\'o finalmente conseguir su t\'itulo
de ingeniero. Luego de otros tantos a\~nos ejerciendo sus estudios (¿?), consigui\'o
ponerse a la cabeza de un gran proyecto de construcci\'on de carreteras.\\

La zona en la que debe trabajar tiene $n$ ciudades con $m$ posibles carreteras
a construir entre ellas. Cada ciudad que sea incluida en el proyecto aportar\'a $a_i$
d\'olares al proyecto, mientras que cada carretera tiene un costo de $w_i$ d\'olares.
Si una carretera se incluye en el proyecto, las ciudades unidas por esta tambi\'en
deben incluirse.\\

El problema estar\'ia en que Tito quiere utilizar una de las habilidades que
aprendi\'o en sus a\~nos de estudio, la de la malversaci\'on de fondos. Todo el dinero
necesario para el proyecto que no sea un aporte de alguna ciudad, lo proveer\'ia
el pa\'is y pasar\'ia por manos de Tito. El dinero aportado por las ciudades no
pasar\'ia por sus manos. Tito quiere maximizar la cantidad de dinero que pasa
por \'el, para poder hacer su magia. Ayude a Tito a seleccionar el conjunto de
carreteras a incluir en el proyecto para lograr su objetivo

\newpage
\begin{abstract}
 Se tienen $n$ ciudades y $m$ posibles carreteras a construir, cada ciudad que sea incluida en el proyecto aportar\'a $a_i$ d\'olares y la construcci\'on de la carretera cuesta $w_i$ d\'olares. Todo el dinero necesario para el proyecto que no sea un aporte de alguna ciudad, lo proveer\'ia el pa\'is y entonces pasar\'ia a manos de Tito. Se quiere seleccionar un conjunto de carreteras a incluir en el proyecto que maximice la cantidad de dinero que pasa
por Tito. 

\end{abstract}

\section{Soluci\'on}

Tenemos que elegir un subconjunto de carreteras y un subconjunto de ciudades, de modo tal que si se elige una carretera, tambi\'en se elijan todas las ciudades de los que depende este proyecto, y tenemos que maximizar la diferencia entre la suma de los costos de las carreteras elegidas(cantidad de dinero que pasa por Tito) y la suma de los aportes de las ciudades unidas a estas(cantidad de dinero que no pasa por Tito), para as\'i maximizar la cantidad de dinero que pasa por Tito.\\

Es bastante inconveniente que por los carreteras obtengamos dinero y por las ciudades que est\'an unidas a esta se pierdan ganancias, as\'i que supongamos que obtenemos dinero del pa\'is por adelantado  por las carreteras y que tenemos que devolver el dinero, si construirla no nos genera ganancias. Estas 2 formulaciones son claramente equivalentes, pero la \'ultima es m\'as f\'acil de modelar.\\

Una forma sencilla ser\'ia crear un grafo con n+m+2 v\'ertices, donde cada v\'ertice representa la fuente(S), el sumidero(T), una carretera o una ciudad. Notaremos a la i-\'esima carretera como $W_i$ y la i-\'esima ciudad como $A_i$. Luego, se agregan aristas de peso $w_i$ entre la fuente y la i-\'esima carretera, aristas de peso $a_i$ entre la i-\'esima ciudad y el receptor, como el dinero que pasa por Tito de las carreteras dependen de las ciudades a las que est\'a unida, para representar esta restricci\'on se agrega una arista de peso $\infty$ entre la carretera y las ciudades que est\'an unidas a esta. Por \'ultimo aplicar un corte m\'inimo sobre este grafo.\\

Si se corta la arista entre la fuente y la i-\'esima carretera, entonces tenemos que devolver el dinero para la i-\'esima carretera. Si se corta la arista entre la i-\'esima ciudad y el receptor, entonces podemos valorar la construcci\'on de las carreteras unidas a la i-\'esima ciudad. Es f\'acil observar que para todas las dependencias entre una carretera y una ciudad, uniremos las ciudades necesarias o abandonaremos el proyecto ya que ser\'ia imposible cortar la arista de peso $\infty$ entre ellos.\\

Luego de aplicar el flujo m\'aximo con el corte antes mencionado nos quedar\'iamos con una red en donde las aristas que est\'an presentes representan las aristas no saturadas, y las que est\'an conectadas al receptor ser\'ian las ciudades cuyo aporte es mayor que el costo de la carretera a la que est\'a unida y por tanto la contrucci\'on de esta no le permitir\'ia a Tito lograr su objetivo. Entonces el conjunto de soluci\'on ser\'ia el de las carreteras que no est\'en unidas a estas ciudades.

\section{Correctitud}

Como problema requiere que se seleccione un conjunto de carreteras y ciudades unidas a estas de tal manera que nos quede el máximo número de beneficios.
Digamos que inicialmente tenemos todos los ingresos de los proyectos. Así que nuestro ingreso total (inicialmente) es igual a $\sum_{i=1}R(w_i)$.

Ahora, como cuando elegimos una carretera, se unen las ciudades correspondientes. Supongamos que las carreteras que no hemos seleccionado están en un conjunto P y las ciudades que nos hacen perder dinero están en un conjunto A. Entonces, definitivamente podemos escribir algo como esto:\\
max\{ingresos\} = $\sum_iR(w_i)-\sum_iR(Pi)-\sum_iC(Ai)$.\\

Podemos ver que la primera de las tres sumas en el lado derecho de la ecuaci\'on es constante. Ahora, si podemos minimizar las dos \'ultimas sumas, podemos obtener los ingresos m\'aximos.\\

Entonces queremos minimizar $\sum_iR(Pi)+\sum_iC(Ai)$.\\ 

As\'i que construimos un grafo donde cada carretera est\'a conectado a la fuente S con capacidad $R(w_i)$ y cada ciudad est\'a conectada al receptor T con capacidad $C(a_i)$. Y como las carreteras dependen de las ciudades que esta\'an unidas a esta, agregamos aristas desde las carreteras hacia las ciudades de las que depende con capacidad $\infty$ para garantizar que si se toma una carretera, también se toman las ciudades requeridas correspondientes.

Finalmente, calculamos el corte m\'inimo del grafo construido.\\

El corte sobre la red de flujo $G$ con flujo m\'aximo $f^*$ es de capacidad m\'inima.
	\textbf{Demostraci\'on:}
	\begin{itemize} 
		\item[]  Sea el corte $(S,T)$ formado por $S = \{s,v_1,v_2,...,v_k\} $ y $T = \{v_{k+1},v_{k+2},...,v_{n-2},t\} $ donde todo v\'ertice $v_i \in S$ es alcanzable por $s$ a trav\'es de un camino en la red residual $G_f$ y todo v\'ertice en $T$ ser\'ian los no alcanzables por $s$ bajo este criterio. Ambos conjuntos son no vac\'ios, pues a $S$ al menos pertenece el v\'ertice $s$ y a $T$ pertenece el v\'ertice $t$, pues si no perteneciera significa que existe un camino de $s$ hacia $t$ en $G_f$ lo que implica que ser\'ia un camino aumentativo y ser\'ia posible incrementar el flujo utilizando dicho camino lo cual entra en contradicci\'on con el hecho de que $f^*$ es un flujo m\'aximo en $G$.
		
		Luego, toda arista $< u,v >$ que cruza el corte $(S,T)$ de $S$ hacia $T$, donde $u \in S$ y $v \in T$ cumple que $f(< u,v >) = c(< u,v >)$, pues si existiese $< u,v >$ que cruza el corte tal que $f(< u,v >) < c(< u,v >)$ entonces la arista $< u,v >$ no estar\'ia saturada y existiera en $G_f$ por lo que el camino de $s$ hacia $u$ de aristas no saturadas, adicionando la arista $< u,v >$ forman un camino $s$ hacia $v$ en $G_f$ lo cual es una contradicci\'on pues $v \in T$. 
		
		Adem\'as toda arista $< v,u >$ que cruza el corte desde $T$ hacia $S$ donde $u \in S$ y $v \in T$ cumple que $f(< u,v >) = 0$, pues si existiese $< v,u >$ que cruza el corte tal que $f(< v,u >) > 0$ entonces dicha arista da origen a una arista inversa $< u,v >$ en $G_f$, como $u \in S$, entonces existe un camino desde $s$ hacia $u$ en $G_f$ que al unirse con la arista $< u,v >$ dar\'ia lugar a un camino de $s$ hacia $v$ en $G_f$ lo cual es una contradicci\'on con el hecho de que $v \in T$.
		
		Por lo tanto todas las aristas que salen de $S$ est\'an saturadas y todas las que entran tienen flujo 0. Demostremos entonces que la capacidad del corte $(S,T)$ es igual al valor del flujo m\'aximo $f^*$ en $G$. Sabemos que $|f^*| = f(S,T)$ por el $Lema$ 26.4 visto en conferencia. Luego: 
		
		$|f^*| = f(S,T) = \sum\limits_{u\in S}\sum\limits_{v\in T}f(u,v)-\sum\limits_{u\in S}\sum\limits_{v\in T}f(v,u)$ 
		
		Pero como demostramos anteriormente que $f(u,v)=c(u,v)$ y $f(v,u)=0$ donde $u \in S$ y $v \in T$ entonces:
		
		$|f^*| = f(S,T) = \sum\limits_{u\in S}\sum\limits_{v\in T}c(u,v)-\sum\limits_{u\in S}\sum\limits_{v\in T}0 = \sum\limits_{u\in S}\sum\limits_{v\in T}c(u,v) = C(S,T)$
		
		Luego la capacidad del corte $(S,T)$ es igual al valor del flujo m\'aximo $f^*$ en $G$ por lo que su capacidad es m\'inima ya que $|f^*|$ es una cota m\'inima de la capacidad de todo corte por $Corolario$ 26.5 visto en conferencias en EDA.  
	\end{itemize}   
	
Para hayar el flujo m\'aximo con corte m\'inimo se utiliza el algoritmo Edmonds-Karp que genera una red residual en donde las aristas que est\'an presentes representan las aristas no saturadas, y las que est\'an conectadas al receptor ser\'ian las ciudades cuyo aporte es mayor que el costo de la carretera a la que est\'a unida y por tanto la contrucci\'on de esta no le permitir\'ia a Tito lograr su objetivo. Entonces el conjunto de soluci\'on ser\'ia el de las carreteras que no est\'en unidas a estas ciudades.\\

\newpage
\section{Complejidad Temporal}
La soluci\'on del problema utiliza el algoritmo Edmonds-Karp para obtener la red de flujo m\'aximo con corte m\'inimo. 

El algoritmo Edmonds-Karp se ejecuta $O(|V|*|E|)$ veces (ver demostraci\'on en I. to A. 3ra. Edición Cap. 26)
Pero sabemos adem\'as, que cada iteraci\'on toma tiempo $ O(|E|)$ ya que el camino aumentativo se halla utilizando un BFS (BFS es $O(|V|+|E|)$ pero en la red de flujo, $|V|$ es $O(|E|)$). El resto de las operaciones, o sea actualizar la Red Residual y el flujo que se va calculando, se pueden hacer tambi\'en en ese mismo orden de tiempo
Por tanto,
el tiempo de ejecuci\'on del algoritmo es $O(|V|*{|E|}^2)$.

Para obtener la soluci\'on luego de aplicar el algoritmo Edmonds-Karp se realiza un ciclo por la cantidad de v\'ertices adyacentes a la fuente en la red de flujo, que en el grafo principal representa la cantidad de carreteras(aristas) por tanto du tiempo de ejecuci\'on es $O(|E|)$

Luego por cada v\'ertice adyacente al receptor que representa las ciudades en la red de flujo se buscan sus v\'ertices adyacentes que son las carreteras para marcarlos como carreteras no v\'alidas a construir, es decir no forma parte de la soluci\'on. Por tanto el tiempo de ejecuci\'on es la cantidad de ciudades por la cantidad de carreteras  $O(|V|*|E|)$

El \'ultimo ciclo es un ciclo en la cantidad de carreteras por tanto es $O(|E|)$

Tambi\'en se utilizaron m\'etodos auxiliares en la construcci\'on del grafo y la matriz adjunta como $ady list$ cuyo costo es $O(|V|*|E|)$, $convert to flow$ con costo de $O(|E|)$.\\

Luego por regla de la suma el costo de la soluci\'on del problema es $O(|V|*{|E|}^2)$.

\newpage
\section{Tester}
\begin{lstlisting}
def tester_solution(grafo, edges_to_build, cities_to_build, pos):

1    if pos == len(grafo.edges):
2        presp = Obtener_presupuesto(grafo, edges_to_build, cities_to_build)
3        return presp

4    max_pres = 0

5    for i in range(pos, len(grafo.edges)):

6        visitado1= False
7        visitado2 = False
        

8        pres_1 = tester_solution(grafo, edges_to_build, cities_to_build, i+1)

9        edges_to_build[i] = 1

10        vert1 = grafo.edges[i].node1
11        vert2 = grafo.edges[i].node2

12        if not cities_to_build[vert1.name] == 1:
13            visitado1 = True
14            cities_to_build[vert1.name] = 1
15        if not cities_to_build[vert2.name] == 1:
16            visitado2 = True
17            cities_to_build[vert2.name] = 1

18        pres_2 = tester_solution(grafo, edges_to_build, cities_to_build, i+1)

19        edges_to_build[i] = 0

20        if visitado1:
21            cities_to_build[vert1.name] = 0
22        if visitado2:
23            cities_to_build[vert2.name] = 0

24        if max(pres_1, pres_2) > max_pres:
25            max_pres = max(pres_1, pres_2)

26    return max_pres
\end{lstlisting}
El m\'etodo tiene como par\'ametro de entrada el grafo, una lista donde se encuentran las carreteras a construir que se representan a trav\'es de las aristas, una lista con los v\'ertices del grafo a construir que representan las ciudades y un entero que representa la posici\'on actual.\\
Lineas 1-2: es la condici\'on de parada del tester\_solution, es decir si la posici\'on actual llego al final de la lista de las aristas del grafo significa que tenemos una soluci\'on. Se calcula el presupuesto teniendo en cuenta las ciudades y las carreteras que se van a construir que est\'an en las listas cities\_to\_build y edges\_to\_build respectivamente y devuelve ese valor.\\
Linea 4: se incializa la variable max\_pres que representa el presupuesto m\'aximo que se tiene hasta el momento.\\
Lineas 5-25: la idea es por cada arista del grafo se quiere hallar el m\'aximo presupuesto entre el presupuesto que se obtiene al construir una arista(carretera) y el presupuesto que se obtiene al no construirla, teniendo en cuenta que construir una arista lleva impl\'icito la construcci\'on de los v\'ertices(ciudades) que est\'an conectados a ella.\\
Linea 8: se realiza la llamada al m\'etodo para construir una soluci\'on en la que la arista actual(carretera) no forma parte de la listas de carreteras a construir, el valor que se obtiene que representa el presupuesto se guarda en la variable pres\_1.\\
Linea 9: se marca la arista actual en la lista de carreteras a construir.\\
Lineas 10-11: Se guardan los v\'ertices que est\'an conectados a la arista actual para marcarlos como v\'ertices(ciudades) a construir.\\
Linea 18: se hace el llamado para hallar el presupuesto teniendo en cuenta que la arista actual forma parte de las aristas a construir y se guarda en la varible pres\_2.\\
Lineas 24: Se compara el m\'aximo presupuesto entre el pres\_1 y el pres\_2 y el presupuesto m\'aximo obtenido hasta el momento.\\
Linea 26: se retorna el presupuesto m\'aximo.\\

\end{document}
